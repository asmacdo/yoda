\documentclass{article}
\usepackage{graphicx} % Required for inserting images

\title{YODA et al (YODA, VAMP): Four pillars of idiomatic version control}
\author{Austin Macdonald}
\date{March 2025}

\begin{document}

\maketitle

\section{Immediate TODOs}

- do research on related compositional patterns across standards and fields

\section{Abstract}

% TODO: compose later when other stuff filled out

\section{Introduction}

Managing data in modern scientific research presents unprecedented challenges due to rapidly expanding dataset scale, complexity, and interdependency.
A workflow is the formal specification of data flow and execution control between components, which include executable code, configuration files, data (inputs and outputs), and provenance records.
Executing a workflow with specific inputs and parameters generates a workflow run—a complete record of the execution process documenting the method, intermediate steps, and outputs.
Currently, however, workflow components—data, code, provenance, and computational environments—are frequently managed separately; this separation introduces complexity and undermines reproducibility, transparency, and rigor.
While version control is a standard practice for code and containers, data versioning and integrated provenance tracking remain frequently overlooked.

The FAIR (Findable, Accessible, Interoperable, Reusable) principles and FAIR4RS guidelines for research software provide structured guidance to ensure that digital objects are managed in ways that support reuse and reproducibility.
The Workflow Community Initiative’s FAIR Workflows Working Group (WCI-FW) explicitly chose not to define new principles specifically tailored for workflows.
Instead, recognizing workflows as hybrid objects that share characteristics of both data and software, they focus on applying established FAIR guidelines directly.

Building on this foundation, YODA offers a set of best practices for structuring the full research objects in a FAIR-aligned manner which would reduce cognitive load in identifying all relevant for a research object components and facilitate reproducibility.

% In this manuscript we will not aim to achieve the ultimate reproducibility of the results, **but** rather provide methodology following which reproducibility could be drastically improved or in many cases indeed lead to full reproducibility even on relatively complex usage scenarios.

By promoting idiomatic version control and modular project organization, YODA Principles helps researchers treat analysis specifications and outputs as composable, structured digital objects.
This approach complements formal computational workflow guidelines (e.g., FAIR4RS, RO-Crate) and facilitates the transition toward comprehensive workflow automation.

% Composition remains "an issue"
% Often relates to "provenance"
% **Introduce existing compositional patterns (references/urls), frictionless data, BIDS (bids uri, SourceDatasets, code/), other approaches to describe; Actionability: PROV **
% Introduce YODA (and VAMP) as attempt to formalize principles on HOW components should be organized using version control system

% Special accent should be pointed to "pragmatism" -- there could be "migrations" between pragmatic and "standardized" (e.g. DataLad run records -> PROV)

\section{Results}

\subsection{Version Control Everything}

Outline:
  - problems with non-version controlled data
  - how version control addresses the problems
    ie, version control tools associate a unique identifier and basic provenance with each revision, and thus enable the identification of precise version states of digital files or collections of files.
  - how traditional version control (git) can be extended to large datasets
  - other benefits
    - collaboration
    - reversion of changes
    - time travel

Research data frequently exists outside the scope of established version control practices, leading to a multitude of reproducibility issues.
Non-version-controlled data creates ambiguity, as precise identification of dataset states is challenging, and historical changes are difficult to reconstruct or verify.
This ambiguity undermines trust, complicates collaboration, and impairs reproducibility, particularly when datasets undergo frequent updates or transformations.

Version control systems (VCS), such as git, directly address these issues by associating each revision with a unique identifier and recording basic provenance information, including who made the changes, when, and why.
Although commonly called VCS, the primary importance of such systems is not in keeping track of "version" identifiers of the components (as "version 1" vs subsequent "version 2"), but rather providing an unambiguous identification and access to the overall state of the content.
For that reason, git describes itself as "the stupid content tracker" (from `man git`) and has content checksum as its underlying content identification mechanism.
Explicit versioning of the states via tags is just a convenience to provide additional semantic on top of different versions of the content.
Throughout the manuscript we will refer to such content tracking systems as VCS to use terminology familiar to the reader, although some initiatives such as \cite{sciops-paper} would describe such requirement as data consistency.

By applying VCS systematically to data and analysis outputs, researchers can pinpoint exact states of digital files, reliably reverting or exploring previous versions as needed.
Although traditional version control systems have limitations when managing large datasets due to storage inefficiencies, modern tools such as git-annex, DataLad, or DVC effectively extend git functionality to handle large, binary, and distributed datasets.
These tools enable researchers to manage extensive datasets without losing the benefits of traditional version control, including improved collaboration through explicit tracking of changes, efficient reversion of undesired modifications, and the ability to reconstruct past research states—effectively enabling "time travel."

Critically, approaching full reproducibility requires that version control encompasses all components essential for replicating computational workflows.
These components typically include raw and derived datasets, preprocessing and analysis scripts, parameter and configuration files, computational environment definitions (such as Docker or Singularity containers), and accompanying documentation describing workflow execution details.
Omitting even one component can severely compromise reproducibility, making comprehensive version control across these assets a fundamental necessity.

\textbf{Formal Principles:}
\begin{itemize}
    \item \textbf{P1.1} All assets essential to replicate computational execution \textbf{MUST} be \emph{included}
    \item \textbf{P1.2} All assets essential to replicate computational execution \textbf{SHOULD} be version controlled using the same version control system
\end{itemize}

% that's where might be worth already to talk about ability of the used VCS/platform to "replicate" the outside components within it.
% e.g. if containers are used, it must provide storage for those containers as part of the resource, and not to rely on external links which might disappear.
% To a degree - VCS should provide ways to "cauterize the wound" (or find a better analogie).
% It should make it "safe to make a cut", but also to "re-connect" the component if any detail or functionality of it is needed.
% And that should lead into modularity + git modules as the way "to cauterize"

\subsection{Modularity}

% TODO: give a different flow into this. Rather accent that research project encompases multiple stages from conception, to raw data collection, analytics, ... publication.
% For reproducibility sake we need to be able to link all of them, but all of them are also useful on their own.
% So similarly to software development where we build a product, which might need compilation from sources - we provide compiled version while not shipping sources + compiler + environment along since they are not needed to use that product.
% Establishing reliable and reproducible builds important and thus done also in software distributions, e.g. see https://wiki.debian.org/ReproducibleBuilds .
% Another factor is that reused components, better exist in single reusable copy to ease fixing/maintenance/reduce size, so in software we e.g. have dynamic libraries.
% So, the boundaries between "modules" are pragmatically driven by establishing "sensibly useful independently module", often, most frequently - derived products adding their unique aspects on top of their "sources".
% Moreover, dealing with such different levels of data processing and modularization facilitates "handoffs" between team members (SciOps).
Monolithic structures present significant barriers to effective development, maintenance, and reuse.
Managing datasets as single, large units complicates incremental updates, obscures dependencies, inhibits flexible reuse or remixing of individual components, and can be notably time-consuming.
The lack of granularity often results in inefficient storage and processing, as small changes may necessitate duplicating or regenerating large portions of data.

% And humans, researchers included, **already** modularize their stuff - They use folders!
% But they often lack guidance on such folders organization and logistics to facilitate efficient project management and collaboration!

Adopting a modular organization, as advocated by YODA, directly addresses these challenges through explicit separation of concerns.
Rather than managing the dataset as one indivisible whole, YODA promotes a compositional approach, structuring projects as assemblies of independently versioned and well-defined components.
Individual parts—such as input and reference datasets, processing scripts, and computational environments—can be updated or replaced independently, thus minimizing disruption and maximizing reusability.

The layout of these components plays a crucial role in enabling effective modularity.
An idiomatic YODA dataset clearly delineates elements into structured directories, such as:

\begin{itemize}
  \item \textbf{code/}: containing analysis scripts and tests
  \item \textbf{inputs/}: encapsulating raw or intermediate datasets
  \item \textbf{envs/}: defining computational environments (e.g., Singularity containers)
  \item \textbf{docs/}: for project documentation
  \item \textbf{results/}: explicitly identified outputs (e.g., figures, tables, derived datasets)
\end{itemize}

% The exact organization might vary depending on the field of investigation and existing conventions.
This intentional organization clarifies how components interact and supports domain-specific standards (e.g., BIDS).
It enables researchers to compose and recombine modules flexibly to produce varied outcomes.

% It is crucial to note that in such idiomatic YODA dataset, all necessary components (such as inputs/ and envs/) necessary to produce results MUST be available within this dataset.
% Referencing materials outside of the folder (e.g. from ../ parent folder, or from online docker hub) breaks the containment and makes module, if defined by the folder boundary, either not portable or loosing guarantees for completeness. 
% DO NOT LOOK UP YODA!

Thus, a dataset version can be precisely described as the pairing of generated results with composition of specific module versions, each independently managed.

% A folder by itself is insufficient to be a "module".
% It requires metadata which would at least demarkate module boundary and potentially associate with specific version control mechanisms and distributions.
% In case of use of git VCS, git repository boundary establishes a module which could be independently re-used while retaining unambigous versioning information across such reuses. 

\textbf{Formal Principles:}

\begin{itemize}
  \item \textbf{P2.1} Assets \textbf{SHOULD} be organized in a modular structure.
  \item \textbf{P2.2} All assets essential to replicate computational execution \textbf{MAY} be included directly in the dataset or linked as subdatasets.
  \item \textbf{P2.3} Components \textbf{SHOULD} accommodate domain-specific standards where applicable.
  % \item \textbf{P2.?} Module boundaries \textbf{SHOULD} be decided on the basis of their \emph{pragmatic idependent usefulness} unless driven by other factors demanding even further modularization.
\end{itemize}

\subsection{Portable Computational Environments}

Research reproducibility fundamentally depends not only on preserving code and data,
but also on capturing the complete computational environment—the operating system,
libraries, dependencies, and tool versions—that produced the results.

Environmental drift, where software dependencies evolve or disappear over time,
is a primary cause of computational irreproducibility. Even with identical code and
data, differences in library versions, compiler settings, or system configurations
can produce divergent results or outright failures.

YODA addresses this through explicit environment specification and versioning.
Computational environments should be:
\begin{itemize}
  \item Explicitly defined (not implicitly assumed)
  \item Machine-reproducible (via containers, Nix/Guix, or language-specific managers)
  \item Version controlled alongside code and data
  \item Self-contained within the project boundary
\end{itemize}

Two primary approaches exist for environment portability:
\begin{enumerate}
  \item \textbf{Container-based}: Docker, Singularity/Apptainer provide OS-level isolation
  \item \textbf{Package-based}: Nix, Guix provide declarative, reproducible package management
\end{enumerate}

YODA principles are environment-mechanism-agnostic; what matters is that
environments are explicitly specified, versioned, and available within the project.

\textbf{Formal Principles:}

\begin{itemize}
    \item \textbf{P4.1} Computational environments \textbf{MUST} be explicitly specified
    \item \textbf{P4.2} Environment specifications \textbf{SHOULD} be machine-reproducible
    \item \textbf{P4.3} Environment definitions \textbf{MUST} be version controlled
    \item \textbf{P4.4} Environments \textbf{SHOULD} be self-contained within the dataset
\end{itemize}

\subsection{Incorporating Provenance into VCS History}

% TODO : cite Tek Raj paper, and paper I found on various provenance formats/approaches.
Provenance—the detailed record of actions taken to produce research outputs—is essential for reproducibility, transparency, and trust in scientific workflows.
Typically, provenance collection occurs separately from version control, leading to fragmentation of essential metadata and difficulties reconstructing workflows accurately or automatically.

YODA advocates embedding provenance directly into version control histories by encoding precise descriptions of computational actions into commit records.
Provenance of all modifications to assets need to be explicitly annotated within these version control histories.
Provenance-rich commit histories enable exact repetition of computational steps or targeted remixing of workflows by selectively modifying certain components.
For modifications driven by code, provenance should be annotated programmatically, and this automated capture requires explicit version information for all components used.

This methodology provides immediate practical utility even before comprehensive workflow automation is achievable, facilitating reproducible research and iterative experimentation.
Furthermore, integrating automated provenance capture into the commit process ensures transparency and reduces human error or oversight, creating inherently detailed, auditable records of the research process.
Collectively, implementing a structured, idiomatic version control strategy using YODA enhances the self-containment and completeness of research workflows, ensures clear, repeatable provenance, and facilitates collaborative, modular, and portable science.

\textbf{Formal Principles:}

\begin{itemize}
    \item \textbf{P3.1} Provenance of all modifications to the assets \textbf{MUST} be annotated.
    \item \textbf{P3.2} Provenance of code-driven modifications to assets \textbf{SHOULD} be annotated programmatically, and \textbf{MUST} include the versions of all components used.
\end{itemize}

\section{Discussion}
\section{Methods}
\section{Data Availability}
\section{Code Availability}
\section{References}

Might be worth relating to some of the principles among "Ten simple rules for principled simulation modelling"
https://journals.plos.org/ploscompbiol/article?id=10.1371/journal.pcbi.1009917 e.g.
"Rule 5: Be kind to your future self (and your readers), and minimise opportunities for errors" since sticking to
YODA would assist in accessing all necessary components and history/reasoning for changes for the "future self".

\section{Author Contributions}
\section{Competing Interests}
\section{Acknowledgments}


\end{document}
