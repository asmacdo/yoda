\documentclass{article}
\usepackage{graphicx} % Required for inserting images

\title{YODA et al (YODA, VAMP): Four pillars of idiomatic version control}
\author{Austin Macdonald}
\date{March 2025}

\begin{document}

\maketitle

\section{Immediate TODOs}

- do research on related compositional patterns across standards and fields

\section{Abstract}

% TODO: compose later when other stuff filled out

\section{Introduction}

Managing data in modern scientific research presents unprecedented challenges due to rapidly expanding dataset scale, complexity, and interdependency.
A workflow is the formal specification of data flow and execution control between components, which include executable code, configuration files, data (inputs and outputs), and provenance records.
Executing a workflow with specific inputs and parameters generates a workflow run—a complete record of the execution process documenting the method, intermediate steps, and outputs.
Currently, however, workflow components—data, code, provenance, and computational environments—are frequently managed separately; this separation introduces complexity and undermines reproducibility, transparency, and rigor.
While version control is standard practice for code and containers, data versioning and integrated provenance tracking remain frequently overlooked.

The FAIR (Findable, Accessible, Interoperable, Reusable) principles and FAIR4RS guidelines for research software provide structured guidance to ensure that digital objects are managed in ways that support reuse and reproducibility.
The Workflow Community Initiative’s FAIR Workflows Working Group (WCI-FW) explicitly chose not to define new principles specifically tailored for workflows.
Instead, recognizing workflows as hybrid objects that share characteristics of both data and software, they focus on applying established FAIR guidelines directly.

Building on this foundation, YODA offers a set of best practices for structuring the full research objects in a reproducible and FAIR-aligned manner.
By promoting idiomatic version control and modular project organization, YODA helps researchers treat analysis specifications and outputs as composable, structured digital objects.
This approach complements formal workflow standards (e.g., FAIR4RS, RO-Crate) and facilitates the transition toward comprehensive workflow automation.

% Composition remains "an issue"
% Often relates to "provenance"
% **Introduce existing compositional patterns (references/urls), frictionless data, BIDS (bids uri, SourceDatasets, code/), other approaches to describe; Actionability: PROV **
% Introduce YODA (and VAMP) as attempt to formalize principles on HOW components should be organized using version control system

% Special accent should be pointed to "pragmatism" -- there could be "migrations" between pragmatic and "standardized" (e.g. DataLad run records -> PROV)

\section{Results}

\subsection{Version Control Everything}

Outline:
  - problems with non-versioncontrolled data
  - how version control addresses the problems
    ie, version control tools associate a unique identifier and basic provenance with each revision, and thus enable the identification of precise version states of digital files or collections of files.
  - how traditional version control (git) can be extended to large datasets
  - other benefits
    - collaboration
    - reversion of changes
    - time travel

Research data frequently exists outside the scope of established version control practices, leading to a multitude of reproducibility issues.
Non-version-controlled data creates ambiguity, as precise identification of dataset states is challenging, and historical changes are difficult to reconstruct or verify.
This ambiguity undermines trust, complicates collaboration, and impairs reproducibility, particularly when datasets undergo frequent updates or transformations.

Version control systems (VCS), such as git, directly address these issues by associating each revision with a unique identifier and recording basic provenance information, including who made the changes, when, and why.
By applying VCS systematically to data and analysis outputs, researchers can pinpoint exact states of digital files, reliably reverting or exploring previous versions as needed.
Although traditional version control systems have limitations when managing large datasets due to storage inefficiencies, modern tools such as git-annex, DataLad, or DVC effectively extend git-like functionality to handle large, binary, and distributed datasets.
These tools enable researchers to manage extensive datasets without losing the benefits of traditional version control, including improved collaboration through explicit tracking of changes, efficient reversion of undesired modifications, and the ability to reconstruct past research states—effectively enabling "time travel."

Critically, approaching full reproducibility requires that version control encompasses all components essential for replicating computational workflows.
These components typically include raw and derived datasets, preprocessing and analysis scripts, parameter and configuration files, computational environment definitions (such as Docker or Singularity containers), and accompanying documentation describing workflow execution details.
Omitting even one component can severely compromise reproducibility, making comprehensive version control across these assets a fundamental necessity.

P1.1 All assets essential to replicate computational execution MUST be included
P1.2 All assets essential to replicate computational execution SHOULD be version controlled using the same version control system

\section{Modularity}

Monolithic structures present significant barriers to effective development, maintenance, and reuse.
Managing datasets as single, large units complicates incremental updates, obscures dependencies, inhibits flexible reuse or remixing of individual components, and can be notably time-consuming.
The lack of granularity often results in inefficient storage and processing, as small changes may necessitate duplicating or regenerating large portions of data.

Adopting a modular organization, as advocated by YODA, directly addresses these challenges through explicit separation of concerns.
Rather than managing the dataset as one indivisible whole, YODA promotes a compositional approach, structuring projects as assemblies of independently versioned and well-defined components.
Individual parts—such as input and reference datasets, processing scripts, and computational environments—can be updated or replaced independently, thus minimizing disruption and maximizing reusability.

The layout of these components plays a crucial role in enabling effective modularity.
An idiomatic YODA dataset clearly delineates elements into structured directories, such as:

\begin{itemize}
  \item \textbf{code/}: containing analysis scripts and tests
  \item \textbf{inputs/}: encapsulating raw or intermediate datasets
  \item \textbf{envs/}: defining computational environments (e.g., Singularity containers)
  \item \textbf{docs/}: for project documentation
  \item \textbf{results/}: explicitly identified outputs (e.g., figures, tables, derived datasets)
\end{itemize}

This intentional organization clarifies how components interact and supports domain-specific standards (e.g., BIDS).
It enables researchers to compose and recombine modules flexibly to produce varied outcomes.
Thus, a dataset version can be precisely described as the pairing of generated results with composition of specific module versions, each independently managed.

\textbf{Formal Principles:}

\begin{itemize}
  \item Assets \textbf{SHOULD} be organized in a modular structure.
  \item All assets essential to replicate computational execution \textbf{MAY} be included directly in the dataset or linked as subdatasets.
  \item Components \textbf{SHOULD} accommodate domain-specific standards where applicable.
\end{itemize}

\section{Incorporating Provenance into VCS History}

Provenance—the detailed record of actions taken to produce research outputs—is essential for reproducibility, transparency, and trust in scientific workflows.
Typically, provenance collection occurs separately from version control, leading to fragmentation of essential metadata and difficulties reconstructing workflows accurately or automatically.

YODA advocates embedding provenance directly into version control histories by encoding precise descriptions of computational actions into commit records.
Provenance of all modifications to assets need to be explicitly annotated within these version control histories.
Provenance-rich commit histories enable exact repetition of computational steps or targeted remixing of workflows by selectively modifying certain components.
For modifications driven by code, provenance should be annotated programmatically, and this automated capture requires explicit version information for all components used.

This methodology provides immediate practical utility even before comprehensive workflow automation is achievable, facilitating reproducible research and iterative experimentation.
Furthermore, integrating automated provenance capture into the commit process ensures transparency and reduces human error or oversight, creating inherently detailed, auditable records of the research process.
Collectively, implementing a structured, idiomatic version control strategy using YODA enhances the self-containment and completeness of research workflows, ensures clear, repeatable provenance, and facilitates collaborative, modular, and portable science.

\textbf{Formal Principles:}

\begin{itemize}
\item Provenance of all modifications to the assets \textbf{MUST} be annotated.
\item Provenance of code-driven modifications to assets \textbf{SHOULD} be annotated programmatically, and \textbf{MUST} include the versions of all components used.
\end{itemize}

\section{Discussion}
\section{Methods}
\section{Data Availability}
\section{Code Availability}
\section{References}
\section{Author Contributions}
\section{Competing Interests}
\section{Acknowledgments}


\end{document}
