\documentclass{article}
\usepackage{graphicx} % Required for inserting images

\title{YODA et al (YODA, VAMP): Four pillars of idiomatic version control}
\author{Austin Macdonald}
\date{March 2025}

\begin{document}

\maketitle

\section{Immediate TODOs}

- do research on related compositional patterns across standards and fields
- 

\section{Abstract}

% TODO: compose later when other stuff filled out

\section{Introduction}

Managing data in modern scientific research presents unprecedented challenges due to rapidly expanding dataset scale, complexity, and interdependency.
A workflow is the formal specification of data flow and execution control between components, which include executable code, configuration files, data (inputs and outputs), and provenance records.
Executing a workflow with specific inputs and parameters generates a workflow run—a complete record of the execution process documenting the method, intermediate steps, and outputs.
Currently, however, workflow components—data, code, provenance, and computational environments—are frequently managed separately; this separation introduces complexity and undermines reproducibility, transparency, and rigor.
While version control is standard practice for code and containers, data versioning and integrated provenance tracking remain frequently overlooked.

The FAIR (Findable, Accessible, Interoperable, Reusable) principles and FAIR4RS guidelines for research software provide structured guidance to ensure that digital objects are managed in ways that support reuse and reproducibility.
The Workflow Community Initiative’s FAIR Workflows Working Group (WCI-FW) explicitly chose not to define new principles specifically tailored for workflows.
Instead, recognizing workflows as hybrid objects that share characteristics of both data and software, they focus on applying established FAIR guidelines directly.

Building on this foundation, YODA offers a set of best practices for structuring the full research objects in a reproducible and FAIR-aligned manner.
By promoting idiomatic version control and modular project organization, YODA helps researchers treat analysis specifications and outputs as composable, structured digital objects.
This approach complements formal workflow standards (e.g., FAIR4RS, RO-Crate) and facilitates the transition toward comprehensive workflow automation.

% Composition remains "an issue"
% Often relates to "provenance"
% **Introduce existing compositional patterns (references/urls), frictionless data, BIDS (bids uri, SourceDatasets, code/), other approaches to describe; Actionability: PROV ** 
% Introduce YODA (and VAMP) as attempt to formalize principles on HOW components should be organized using version control system

% Special accent should be pointed to "pragmatism" -- there could be "migrations" between pragmatic and "standardized" (e.g. DataLad run records -> PROV)

\section{Results}

% ... pillars ...


Yarik feels that we have

- Composition
 - self containment
 - compute environments 
- Version control
- Actionability (PROV)

\section{Discussion}
\section{Methods}
\section{Data Availability}
\section{Code Availability}
\section{References}
\section{Author Contributions}
\section{Competing Interests}
\section{Acknowledgments}


\end{document}
